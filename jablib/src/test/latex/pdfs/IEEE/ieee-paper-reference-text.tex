\documentclass[conference,a4paper,english]{IEEEtran}[2015/08/26]

\usepackage{lipsum}
\usepackage{hyperref}
\usepackage{pbalance}

\begin{document}
\title{JabRef Example for Parsing Related Work Section}
\author{%
  \IEEEauthorblockN{First Author}
  \IEEEauthorblockA{University of Examples, Germany\\
    \{lastname\}@example.org}
}

\maketitle

\begin{abstract}
\lipsum[1]
\end{abstract}

\section{Introduction}
\lipsum[2]

\section{Related Work}

% Text generated by https://app.scienceos.ai
A study on the Colombian chocolate industry represents the first application of a social life cycle assessment (S-LCA) to cover both cocoa cultivation and chocolate manufacturing \cite{LunaOstos_2024}.
A study on selective, hedonic deprivation found that restricting chocolate intake for two weeks increased state chocolate craving, but only in individuals who already had high trait chocolate craving~\cite{Richard_2017}.
Research suggests that the health benefits of moderate cocoa or dark chocolate consumption, which include cardiovascular and cognitive advantages, likely outweigh the risks associated with its high energy density~\cite{Katz_2011}.

\section{Contribution}
\lipsum[4]

\section{Conclusion and Outlook}
\lipsum[3]

\bibliographystyle{IEEEtran}
\bibliography{IEEEabrv,ieee-paper-reference-text}

\end{document}
